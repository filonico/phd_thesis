% \documentclass[../main.tex]{subfiles}

% \begin{document}

{
\setstretch{1.0}
\chapter{Conclusions}
\label{chapter:conclusions}
}

The main objective of this PhD thesis was to investigate bivalve \gls{sd} through the lens of evolutionary and integrative biology. Bivalves is a group of animals characterised by highly heterogeneous sexual and reproductive modes, with strictly gonochoristic species, obligate and facultative hermaphrodites (either protandrous, protogynous and bidirectional), as well as androgenetic systems. Both genetic and environmental factors seem to influence the sexual identity, at various degrees according to the species, and \glspl{hesc} seem to have not been selected throughout the bivalve evolutionary history. Therefore, a rigorous comparative approach is essential to unravel the extreme complexity that regulates bivalve \gls{sd}. Particularly, by combining bioinformatics with \textit{wet-lab} techniques, including genomics, phylogenetics, molecular evolution analyses, \gls{dge}, mRNA \textit{in-situ} \gls{hcr}, and immunolocalization, this work lays the foundation to understanding how \glspl{srg}, with a special focus on the \gls{dsfg} families, have evolved and may function in sex-determining processes across the bivalve taxonomic diversity.

In \cref{chapter:perspective}, the emerging role of bivalves as model organisms for \gls{sd} studies has been emphasised through a critical examination of the current knowledge. The complexity of bivalve reproduction and sexual systems underscored the need to view \gls{sd} not as a binary and stationary process, but rather as a highly dynamic continuum influenced by multiple genetic and environmental factors. Adopting this broader perspective will allow for a more effective investigation of the biology of \gls{sd}.

In \cref{chapter:molecularEvolution}, the molecular evolution of \glspl{srg} across a range of bivalve species was analysed. The findings revealed patterns of accelerated \gls{aasd} in key \glspl{srg}, namely \gls{dmrt-1l} and \textit{Sox-H}, supporting the hypothesis that these genes are deeply involved in \gls{sd} mechanisms, possibly even as primary \glspl{sdg}. Thanks to a comparative study which encompassed the analysis of additional control datasets---mammals and \textit{Drosophila}, the validity of the results has been confirmed and discussed in the light of a broader framework. This comparative approach allowed for the identification of evolutionary convergences and divergences, advancing our understanding of the patterns of molecular evolution in animal \glspl{srg}.

\cref{chapter:insitu} focused on gene expression studies in the Mediterranean mussel \gls{mgal}, offering insights into the \gls{sd} process in early development. Particularly, \gls{dmrt-1l} and \textit{Sox-H} appear to not be expressed during these stages, while \textit{Fox-L2} transcription starts only with the onset of gastrulation. This suggests that either these genes are not top regulators of \gls{sd}, or that \gls{sd} occurs only later in development, thus their expression is not found during the analysed stages. The latter interpretation would be in line with the pattern of \gls{pgc} specification in \gls{mgal}, which begins only in correspondence with the onset of gastrulation, thus not following a strict preformation model as in other studied bivalves.

Overall, this thesis further demonstrates that bivalves, with their vast reproductive and sexual diversity, serve as ideal models for investigating the complexity of \gls{sd}. By integrating genomic analyses with developmental biology, this work provides a new framework for understanding how \gls{srg} evolve and function in diverse species. Future studies could build on these insights by exploring the functional roles of \glspl{srg} through other advanced techniques (such as CRISPR-Cas9), thus expanding our understanding of the genetic underpinnings of \gls{sd} and differentiation. This integrative approach has the potential to unlock new knowledge not only in bivalves but across a wide array of species, deepening our understanding of the evolutionary forces shaping reproductive biology in animals.

% end{document}