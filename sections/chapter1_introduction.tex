\documentclass[../main.tex]{subfiles}

\begin{document}

{
\setstretch{1.0}
\chapter{Introduction}
\label{introduction}
}

\section{Molecular mechanisms of sex determination and the evolution of sex-determining genes}
The process of sex determination (SD) has been traditionally associated with the very first step of gonad differentiation, where an initial trigger activates the molecular pathway that establishes organism sex. According to this view, two alternative types of SD have been recognized at first: the environmental sex determination (ESD) and the genetic sex determination (GSD), depending on whether the very first cues are of environmental or genetic origin. Conversely, all the downstream events of gonad development (i.e., after SD) have been appointed as primary sex differentiation (PSD), which consists of the entire set of morphogenetic, molecular, and physiological events leading to the full maturation of testes or ovaries (\textbf{\cite{uller2011origin, beukeboom2014evolution}}). Lately, however, the dichotomous views of ESD/GSD and of SD/PSD have been questioned. On the one hand, a growing number of studies on non-model organisms proved that ESD and GSD represent a continuum of mixed conditions rather than two mutually exclusive phenomena. On the other, the high evolutionary dynamics and the variable expression patterns of the genes involved in the processes of gonad commitment and development make the distinction between SD and PSD of unclear utility (\textbf{\cite{beukeboom2014evolution}}).

Considering this complex scenario, \textbf{\cite{uller2011origin}} proposed a unified and broad-scope definition for SD, that is, “the processes within an embryo leading to the formation of differentiated gonads as either testes or ovaries”, without any actual distinction between environmental/genetic initial triggers or the downstream effectors. However, I argue that this definition should be expanded to encompass not only the embryonic stage of the animal life cycle but also adulthood, since cases of sex reversals and sex changes (sequential hermaphroditism) legitimately express proper SD processes during post-embryonic life stages as well.

In its most intimate core, animal SD is the manifestation of complex gene regulatory networks where, in accordance with the Wilkins’ theory (1995), the downstream actors appear to be nearly conserved both from functional and identity point of views, while the master top regulators (the commonly recognized sex determinants, such as the sex-determining region of chromosome Y [SRY] in therians or the ratio between sex and autosome chromosomes in \textit{Drosophila}) are often the most variable part (\textbf{\cite{beukeboom2014evolution}}). As a matter of fact, this evolutionary pattern of animal sex-determining cascades has been observed in major animal clades, including vertebrates (e.g., \textbf{\cite{marshall2010homologies}}), insects (e.g., \textbf{\cite{verhulst2010insect}}), and nematodes (e.g., \textbf{\cite{stothard2003sex}}).

Sex-determination related genes (SRGs) are of particular interest not only from a regulatory point of view but also because of their patterns of molecular evolution. In fact, transcriptionally sex-biased genes (including SRGs) often tend to evolve faster than unbiased genes at the level of protein sequences. In particular, male-biased genes generally show higher rate of sequence evolution in comparison to both female-biased and unbiased counterparts (reviewed in \textbf{\cite{parsch2013evolutionary,grath2016sex}}), as it has been repeatedly observed in well-studied organisms such as fruit flies (e.g., \textbf{\cite{meisel2013faster}}), nematodes (e.g., \textbf{\cite{cutter2005sexual}}), mice (e.g., \textbf{\cite{kousathanas2014faster}}) and primates (e.g., \textbf{\cite{khaitovich2005parallel}}), and in other emerging model systems, such as \textit{Daphnia pulex} (\textbf{\cite{eads2007profiling}}), aphids (\textbf{\cite{purandare2014accelerated}}), and two wasp species of the genus \textit{Nasonia} (\textbf{\cite{wang2015nasonia}}). Growing evidence is however showing cases in which instead female-biased genes have higher rates of sequence evolution than male-biased genes, such as in mosquitoes of the genus \textit{Anopheles} (\textbf{\cite{papa2017anopheles}}), and European and Manila clams of the genus \textit{Ruditapes} (\textbf{\cite{ghiselli2018comparative}}).

The pattern of molecular evolution of sex-biased genes is particularly evident in organisms with sex chromosomes (both in XY/ZW and X0 systems), such as fruit flies, birds and mammals, where the so-called fast-X (or fast-Z) effect has been extensively reported for sex-chromosome associated genes (\textbf{\cite{vicoso2006evolutionXchrom,meisel2013faster,mank2007fastZ}}). This high rate of sequence evolution in sex-biased genes and sex chromosomes (SCs) can be the result of both adaptative and non-adaptative processes, since the observed higher ratio between non-synonymous and synonymous mutations (dN/dS) can be caused by natural selection, sexual selection or sexual antagonism, as well as genetic drift (\textbf{\cite{vicoso2006evolutionXchrom,parsch2013evolutionary,meisel2013faster,grath2016sex}}).

\section{Sex determination in bivalves: a long-standing enigma}
Bivalves are the second largest clade in molluscs, counting more than 18,000 species (\href{https://www.catalogueoflife.org/}{Catalogue of Life}) distributed at all depths and in all marine environments, as well as in some freshwater habitats. Thanks to their high diversity and biological peculiarities, they have been proposed as promising model organisms for investigating a wide array of biological, ecological and evolutionary issues (\textbf{\cite{milani2020faraway,ghiselli2021bivalve}}). However, despite their socio-economic and scientific importance, the knowledge concerning the molecular basis of bivalve reproduction and SD is still quite limited (\textbf{\cite{breton2018sex}}). Clues from various works seem to suggest that both genetic and environmental factors (e.g., temperature, food availability, and steroids) are involved in SD, and that heteromorphic sex chromosomes (HeSCs) are absent (\textbf{\cite{breton2018sex,han2022ancient}}). However, the exact process by which sex is determined and gonad commitment is established is, currently, still unknown. Actually, bivalves represent a dazzling example of how the traditional dichotomies between ESD/GSD and SD/PSD can sometimes hamper scientific research, as many bivalve species exhibit various forms of hermaphroditism and because a master environmental or genetic sex determinant inducing PSD may just not exist.

In the attempt to identify SRGs, many differential gene expression analyses have been recently performed on a variety of species covering most of the phylogenetic diversity of bivalves (e.g., \textbf{\cite{milani2013nuclear,zhang2014genomic,chen2017transcriptome,capt2018deciphering,ghiselli2018comparative,shi2018proteome}}). Some of the genes that were found to be differentially expressed between gonads of different sex were systematically retrieved across species, such as those belonging to the \textit{Dmrt}, \textit{Sox}, and \textit{Fox} families, which act in concert in various animal developmental processes including the SD cascade (\textbf{\cite{marshall2010homologies,beukeboom2014evolution}}). To this regard, \textbf{\cite{zhang2014genomic}} proposed a working model for the sex-determining pathway of the Pacific oyster \textit{Crassostrea gigas} in which: \textit{CgSoxH} promotes male gonad development by activating \textit{CgDsx}, which belong to the \textit{Dmrt} family, and inhibiting \textit{CgFoxL2}; \textit{CgFoxL2}, when not inhibited by the pair \textit{CgSoxH}/\textit{CgDsx}, promotes female gonad development. Moreover, \textbf{\cite{han2022ancient}} recently identified homomorphic sex chromosomes (HoSCs) in eight scallop species and appointed \textit{FoxL2} as a putative SRG in \textit{Patinopecten yessoensis} and \textit{Chlamys farreri}. Though, much of the recent research effort on bivalve SRGs has been limited to their molecular cloning, differential transcription, and tissue localization (\textbf{\cite{liang2019sox2,sun2022examination}}). Furthermore, few works have directly investigated the biological functions of \textit{Dmrt}, \textit{Sox}, and \textit{Fox} genes in bivalves so far, and most used post-transcriptional silencing of target mRNAs (RNA interference [RNAi]). \textbf{\cite{liang2019sox2}} studied the role of \textit{Sox2} in the spermatogenesis of the Zhikong scallop \textit{Chlamys farreri} and found that it likely regulates proliferation of spermatogonia and apoptosis of spermatocytes, since its knockdown resulted in the loss of male germ cells. \textbf{\cite{wang2020identification}} proposed that in the female gonads of the freshwater mussel \textit{Hyriopsis cumingii}, \textit{FoxL2} might be related to the \textit{Wnt}/\textit{ $\beta$-catenin} signaling pathway, which takes part in ovarian differentiation also in vertebrates. \textbf{\cite{sun2022examination}} found instead that in \textit{C. gigas}, \textit{FoxL2} and \textit{Dmrt1L} mRNA knockdown results  in the size reduction of female and male mature gonads, respectively.

In this sense, bivalve molluscs represent a striking example of the difficulty to reconcile the traditional view of a single sex determinant with an apparent multifactorial model in which many genes and environmental cues act in concert to establish the sexual identity of the individual (\textbf{\cite{breton2018sex}}). Lately, much effort has been put in the characterisation of bivalve SD and a general framework is eventually taking shape. Functional assays with RNAi and CRISPR-Cas9 techniques (e.g., \textbf{\cite{wang2020identification,sun2022examination,wang2022transcriptome}}), as well as with \textit{in-situ} localization and immunohistochemistry (e.g., \textbf{\cite{perez2011cytogenetic,milani2013nuclear}}), are making their way into the study of bivalve biology and have been proved essential instruments also for the investigation of sex-related traits. However, very few works have made extensive use of the comparative and integrative approach in bivalve studies so far, which hampers the possibility to infer general patterns for such a vast class of organisms (\textbf{\cite{milani2020faraway}}). The high evolutionary rates and plasticity of SRGs make the situation even harder, since phylogenetic and orthology inferences can lead to erroneous reconstructions in the presence of signal saturation and high sequence divergence (reviewed in \textbf{\cite{natsidis2021systematic,lozano2022practical}}).

\end{document}