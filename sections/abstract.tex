{
\chapter*{Abstract}
\label{abstract}
\addcontentsline{toc}{chapter}{Abstract}
\markboth{Abstract}{Abstract}
}

\Gls{sd} in bivalves remains a largely unexplored field, despite the socio-economic importance of many species. This can be traced back to the diversity of mechanisms observed across species, involving both genetic and environmental factors and apparently lacking heteromorphic sex chromosomes, which hamper a straightforward scientific research. This study presents an integrative approach that combines comparative genomics, phylogenetic analyses, and \textit{in-situ} visualisation techniques to investigate the molecular basis of \gls{sd} in bivalve molluscs. Using a broad phylogenetic and genomic framework, key components of gene families known for their roles in \gls{sd} across Metazoa were identified and analysed through the lens of comparative genomics. Particularly, considering that sex-determining genes tend to evolve faster than genes not involved in \gls{sd}, we leveraged the tools of molecular evolution to identify highly-divergent genes among the Dmrt, Sox, and Fox gene families. Both \textit{Dmrt-1L} and \textit{Sox-H} were found to be included in the group of bivalve fast-evolving genes, giving support to previous works which appointed them as tighlty involved with male \gls{sd} in bivalves. To further investigate the roles of these genes, mRNA \textit{in-situ} \gls{hcr} was employed to look at their transcription patterns during the embryonic and early larval stages of \gls{mgal}, along with expression patterns of \textit{Fox-L2}---a gene previously associated with bivalve female \gls{sd}, and of the germline marker \textit{Vasa}/Vasa. Both \textit{Dmrt-1L} and \textit{Sox-H} were found to be not transcribed during the sampled stages (up until 72 hours post fertilization), while \textit{Fox-L2} showed an increasing sex-unbiased expression with the onset of gastrulation. Therefore, \gls{sd} is likely not happening during these early developmental stages. This observation aligns with the expression of \textit{Vasa}/Vasa, whose specification of \glspl{pgc} seemed to be relying on a mixed process of preformation and epigenesis. Before gastrulation, both \textit{Vasa}/Vasa is homogeneously present in all blastomeres, thus not labelling presumptive \glspl{pgc} univocally. The process of \glspl{pgc} formation seems instead to start after the formation of the larvae, when \textit{Vasa}-positive cells begin to accumulate in two lateral areas at both sides of the larvae. Therefore, \gls{sd} can be expected to occur only after this stage. The present work shows the importance of employing an integrative analysis when investigating overlooked processes in non-model organisms. Particularly, this contributes a foundational reference for \gls{sd} in bivalves, broadening our understanding of the genetic factors shaping reproductive biology in this ecologically and economically significant group.