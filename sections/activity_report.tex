{
\chapter*{Activity report}
\label{activityreport}
% \addcontentsline{toc}{chapter}{Activity report}
\markboth{Activity report}{Activity report}
}

\setcounter{page}{1}

\normalsize
\textit{This is the report of the activities carried out during my 3-year PhD course (2021--2024)}.

\section*{Research activity}
%\normalsize
\textit{Here are the research activities not directly related to the main topic of the PhD thesis.}
\begin{itemize}
    \item Manual curation of long interspersed nuclear element (LINE) libraries of several bivalve species;
    \item comparative genomics analysis of Hox and ParaHox genes in branchiopod crustaceans;
    \item comparative genomics analysis of branchiopod crustaceans to investigate the molecular underpinnnings of morphological stasis and genome size variations;
    \item molecular phylogenetics and Bayesian dating of branchiopod crustaceans;
    \item preparation of mRNA sequences of genes involved in body segmentation in \textit{Triops cancriformis} (Pancrustacea, Branchiopoda), to be used to generate probes for mRNA \textit{in-situ} HCR on larvae (in collaboration with the Patel Lab; Marine Biology Lab, Woods Hole, MA, USA);
    \item collection, fixation, and storing of jouvenile stages of several stick insect (Insecta, Phasmida) species, to be used for mRNA \textit{in-situ} HCR to investigate the temporal and spatial transcription of genes involved in wing morphogenesis (in collaboration with the Patel Lab; Marine Biology Lab, Woods Hole, MA, USA);
    \item preparation of a review on the evolutionary causes and consequences of trait loss reversals;
    \item preparation of mitotic chromosome plates in the red wood ant \textit{Formica paralugubris} from cerebral ganglia of pre-pupae.

\end{itemize}

\section*{Visiting scholar}
\begin{itemize}
    \item Nuzhdin Lab (University of Southern California, Los Angeles, CA, UA; Aug 20, 2023--Feb 20, 2024), to accomplish the abroad period of my PhD;
    \item Juan Pasantes\curlyapostrophe{} lab (University of Vigo, Vigo, Spain; Jan 12--22, 2023), for a specific training on chromosome mitotic plate preparation in bivalve species.
\end{itemize}

\section*{Teaching activity}
\begin{itemize}
    \item Practical class \doublecurlyquotes{CAFE: estimating gene family turnover across a phylogenetic tree} (Apr 23, 2024) for first-year students of the course \doublecurlyquotes{Molecular phylogenetics} pursuing a Master degree in \doublecurlyquotes{Bioinformatics} at the University of Bologna (Italy);
    \item Practical invertebrate zoology class (Sep 2022--Jan 2023) for first-year students pursuing a Bachelor degree in \doublecurlyquotes{Biological Sciences} at the University of Bologna (Italy).
\end{itemize}

\section*{Co-supervised thesis}
\begin{itemize}
    \item \textit{Evaluation of different calibration methods on Branchiopoda (Crustacea) phylogeny}. Niccolò Righetti. Master degree in \doublecurlyquotes{Biodiversità ed evoluzione}, University of Bologna, Bologna (Italy). Supervisor: Andrea Luchetti. Co-supervisor: \underline{Filippo Nicolini}. AA 2022/2023;
    \item \textit{Filogenesi molecolare di alcune famiglie dell\curlyapostrophe ordine Phasmatodea con enfasi sulla famiglia Heteropterygidae (Bacilloidea)}. Giacomo Orsini. Bachelor degree in \doublecurlyquotes{Scienze biologiche}, University of Bologna, Bologna (Italy). Supervisor: Andrea Luchetti. Co-supervisor: Simona Corneti, \underline{Filippo Nicolini}. AA 2021/2022;
    \item \textit{Filogenesi molecolare di specie appartenenti alle famiglie Heteropterygidae e Anisacanthidae (Phasmatodea, Bacilloidea)}. Alessandro Siragusa Camacho. Bachelor degree in \doublecurlyquotes{Scienze biologiche}, University of Bologna, Bologna (Italy). Supervisor: Andrea Luchetti. Co-supervisor: Simona Corneti, \underline{Filippo Nicolini}. AA 2021/2022;
    \item \textit{Filogenesi molecolare di specie della famiglia Pseudophasmatidae}. Giovanni Amedeo Paselli. Bachelor degree in \doublecurlyquotes{Scienze biologiche}, University of Bologna, Bologna (Italy). Supervisor: Barbara Mantovani. Co-supervisors: Simona Corneti, \underline{Filippo Nicolini}. AA 2020/2021.
\end{itemize}

\section*{Courses and workshops}
\begin{itemize}
    \item \textit{Establishing state-of-the-art mollusc genomics}. EMBO Workshop. Namur, Belgium. May 28--31, 2024;
    \item \textit{Art (Science) Attack}. Physalia Courses. Online. May 20--23, 2024;
    \item \textit{Introduction to Python for biologists}. Physalia Courses. Online. Sep 25--28, 2023;
    \item \textit{ITA*PHY phylogenetics workshop}. Trento, Italy. Jun 6--9, 2023;
    \item \textit{Sex chromosome evolution}. Physalia Courses. Online. Jan 23--27, 2023.
\end{itemize}

\section*{Awards and scholarships}
\begin{itemize}
    \item Travel grant to attend the \doublecurlyquotes{Evoluzione2024} congress in Naples (Italy). Stazione Zoologica Anton Dohrn. Sep 8--11, 2024;
    \item Travel grant to attend the EMBO workshop \textit{Establishing state-of-the-art mollusc genomics} in Namur (Belgium). EMBO. May 28--31, 2024;
    \item Laura Bassi scholarship for editorial assistance to postgraduates and junior academics. Editing Press. Apr 13, 2023.
\end{itemize}

\section*{Presentations at congresses}
\subsection*{Oral presentations}
\begin{itemize}
    \item \underline{Nicolini F}, Iannello M, Piccinini G, Ghiselli F, Luchetti A, Milani L. (2024). Advancing the study of bivalve sex determination in the light of comparative genomics. Establishing state-of-the-art mollusc genomics (EMBO workshop). Namur (Belgium). May 27-30, 2024;
    \item \underline{Nicolini F}, Ghiselli F, Milani L, Luchetti A. (2023). Contrasting patterns of amino acid evolution and shared ancestry between putative sex-determining genes in bivalve molluscs. EVOLMAR 2023. Online. Nov 14-17, 2023;
    \item \underline{Nicolini F}, Ghiselli F, Milani L, Luchetti A. (2023). Sex-determination related genes in bivalves: novel acquisitions and high rates of sequence evolution. Evolution 2023 (Ernst Mayr Award symposium). Online. Jun 2-3, 2023.
\end{itemize}

\subsection*{Poster presentations}
\begin{itemize}
    \item \underline{Nicolini F}, Iannello M, Piccinini G, ghiselli F, Nuzhdin S, Luchetti A, Milani L. (2024). How to detect sex-determinig genes through molecular evolution: bivalves a a case study. Evoluzione 2024. Naples, Italy. Sep 8--11, 2024;
    \item \underline{Nicolini F}, Ghiselli F, Milani L, Luchetti A. (2022). Clues of accelerated molecular evolution in gene families associated wit sex determination in bivalves. SMBE 2023. Ferrara, Italy. Jul 24--27, 2023;
    \item \underline{Nicolini F}, Ghiselli F, Milani L, Luchetti A. (2022). Clues of accelerated molecular evolution in gene families associated wit sex determination in bivalves. SIBE/ISEB 2022. Ancona, Italy. Sep 4--7, 2022;
    \item \underline{Nicolini F}, Martelossi J, Forni G, Mantovani B, Luchetti A. (2021) First insights and comparative genomics of Hox and ParaHox genes in tadpole shrimps. EuroEvoDevo 2022. Naples, Italy. May 31--Jun 3, 2022.
\end{itemize}

\subsection*{Invited talks}
\begin{itemize}
    \item \textit{From comparative genomics to fluorescence imaging: a multi-disciplinary approach to study bivalve sex determination}. Auer Lab. University of Fribourg, Fribourg. Jul 26, 2024.
\end{itemize}

\subsection*{Outreach activity}
\begin{itemize}
    \item Editor and web writer for BioPills -- the Italian community of life sciences (\href{https://biopills.net/}{biopills.net/}). Jul 2017--ongoing;
    \item Presenter at the European Researchers\curlyapostrophe{} Night 2024, University of Bologna, Bologna (Italy). Sep 27, 2024;
    \item Presenter at the BiGeA Day 2023, University of Bologna, Bologna (Italy). May 27, 2023;
    \item Opening Days, University of Bologna, Bologna (Italy). Nov 18, 2022.
\end{itemize}

\subsection*{Scientific publications}
\small{\textit{* equal contribution}}
\begin{itemize}
    \item Righetti N*, \underline{Nicolini F*}, Forni G, \& Luchetti A. (2024). Towards a time-tree solution for Branchiopoda diversification: a jackknife assessment of fossil age priors. \textit{Submitted for peer-review}
    \item \underline{Nicolini F}, Ghiselli F, Luchetti A, \& Milani L. (2023). Bivalves as emerging model systems to study the mechanisms and evolution of sex determination: a genomic point of view. \textit{Genome Biology and Evolution}, \textit{15}(10), evad181. doi: \href{https://doi.org/10.1093/gbe/evad181}{10.1093/gbe/evad181}
    \item Martelossi J, \underline{Nicolini F}, Subacchi S, Pasquale D, Ghiselli F, \& Luchetti A. (2023). Multiple and diversified transposon lineages contribute to early and recent bivalve genome evolution. \textit{BMC Biology}, \textit{21}(1), 1--23. doi: \href{https://doi.org/10.1186/s12915-023-01632-z}{10.1186/s12915-023-01632-z}
    \item \underline{Nicolini F}, Martelossi J, Forni G, Savojardo C, Mantovani B, \& Luchetti A. (2023). Comparative genomics of Hox and ParaHox genes among major lineages of Branchiopoda with emphasis on tadpole shrimps. \textit{Frontiers in Ecology and Evolution}, \textit{11}, 23. doi: \href{https://doi.org/10.3389/fevo.2023.1046960}{10.3389/fevo.2023.1046960}
    \item Forni G, Cussigh A, Brock PD, Jones BR, \underline{Nicolini F}, Martelossi J, Luchetti A, \& Mantovani B. (2023). Taxonomic revision of the Australian stick insect genus \textit{Candovia} (Phasmida: Necrosciinae): insight from molecular systematics and species-delimitation approaches. \textit{Zoological Journal of the Linnean Society}, \textit{197}(1), 189--210. doi: \href{https://doi.org/10.1093/zoolinnean/zlac074}{10.1093/zoolinnean/zlac074}
\end{itemize}