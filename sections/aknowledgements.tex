% \documentclass[../main.tex]{subfiles}

% \begin{document}

{
\chapter*{Aknowledgements}
\addcontentsline{toc}{chapter}{Aknowledgements}
\label{aknowledgements}
\markboth{Aknowledgements}{Aknowledgements}
}

\begin{otherlanguage*}{italian}
Questa volta ho come l\curlyapostrophe{}impressione che possa essere un po\curlyapostrophe{} più difficile scrivere i ringraziamenti. Quello di dottorato è stato un percorso decisamente diverso da quello dei due tirocini all\curlyapostrophe{}università, sia per sua costituzione, che per come l\curlyapostrophe{}ho vissuto io. Mi verrebbe naturale dividere i ringraziamenti su più livelli, distinti tra loro ma allo stesso tempo complementari (un po\curlyapostrophe{} come la selezione naturale che agisce su più piani\ldots{} ma questa è un'altra storia): da un lato ci sarebbero i ringraziamenti più \doublecurlyquotes{professionali}, rivolti a coloro che hanno giocato un ruolo diretto nella mia vita universitaria; dall'altro lato, invece, ci sarebbero i ringraziamenti strettamente \doublecurlyquotes{personali}, rivolti alle persone che con il dottorato hanno avuto poco o nulla a che fare. Tuttavia, mi resta difficile pensare ad una distinzione così netta, perché, in una qualche maniera, gran parte delle persone mi è stata d\curlyapostrophe{}aiuto su entrambi i fronti, anche laddove magari non era scontato. Tra le cose buone che ha fatto il dottorato, infatti, c\curlyapostrophe{}è sicuramente quella di avermi fatto crescere in proporzione più al livello personale che accademico. E quindi ogni singola persona richiederebbe dei ringraziamenti piuttosto ampi, il che però mi prenderebbe almeno altre 100 pagine. Proverò ad essere il più sintetico possibile.

% Non l\curlyapostrophe{}avrei mai detto, eppure durante questi 3 anni ho fatto la mia prima esperienza di ansia. E non quell\curlyapostrophe{}ansia da \doublecurlyquotes{Oh mio dio domani è lunedì!} o da \doublecurlyquotes{Chissà se l\curlyapostrophe{}articolo verrà accettato}. No. L\curlyapostrophe{}ansia di cui parlo è stata quella che ti toglie il fiato, che ti fa piangere e che blocca ogni singola parte di te per almeno 2--3 minuti. Quella serie insomma. \textit{E che c\curlyapostrophe{}entra tutto ciò con i ringraziamenti?} direte voi. Beh, c\curlyapostrophe{}entra perché è proprio uno dei motivi per cui sto facendo fatica a scriverli e pensarli, i ringraziamenti.

Nicola sei ovviamente il primo. Che dire. Tante cose, troppe cose. Non saprei da dove cominciare. Sei sicuramente la persona che, per ovvi motivi, in questi anni di dottorato mi ha visto davvero in ogni mood: dall'euforia per la più banale foto al microscopio di una pupa di formica, ai momenti di estraneazione totale quando tornavo a casa da una giornata demoralizzante. Eppure sei sempre stato lì, mai una volta che ti sei spazientito o che mi hai fatto pesare i continui sbalzi d'umore. Neppure quando ti dissi che il locale a Los Angeles in cui Jennifer Lopez stava servendo drink allx clienti era quello in cui ero stato giusto dieci giorni prima (beh, in quell\curlyapostrophe{}occasione forse un po\curlyapostrophe{} ti sei spazientito, sì). Ma comunque c\curlyapostrophe{}eri sempre, lì ad ascoltare le mie paure e i miei progetti, i miei mille ripensamenti e le mie fantasticherie sul riempire casa di piante e crani di animali. Ti ho già detto di sì una volta, e te lo ridirei altre mille.

Il secondo ringraziamento non può che andare a mia madre, mio padre e mio fratello. Anche se a volte non capivate cosa studiassi o perché lo facessi, non avete mai smesso di sostenermi. Sicuramente è più semplice (e forse fa più effetto) poter dire alle persone \doublecurlyquotes{Ah! Mio figlio/fratello studia il cancro!}. Ma no, vi tocca dire che ho studiato come fanno sesso le vongole e al limite poi buttarla in caciara perché vi dico che in fondo le cozze e i moscioli sono la stessa cosa, solo che gli anconetani se la tirano un po'. So di non essere una persona che condivide troppe cose di sé o della sua vita quotidiana (forse sto vivendo la mia adolescenza in ritardo), ma sono sicuro che spesso, durate i miei silenzi, avete capito più cose di quanto immaginassi. Vi voglio bene.

Prof Luchetti, a lei l'onore di chiudere la top 3. In questi anni la sua spensierata pragmaticità è riuscita a farmi districare dagli intrecci dell'accademia, ma soprattutto dagli intrecci che mi creavo da solo. Sento di dovermi ripetere: so di non essere una persona che condivide troppe cose di sé o dei suoi progressi, ma in qualche modo lei ha sempre avuto la parola giusta. Anche quando forse nessuno dei due se ne è reso conto. La lezione più importante che mi ha dato? \doublecurlyquotes{Non aprire mai un barattolo di fagioli con una ruspa} (eppure eccomi qui a scrivere in \LaTeX{} anche i ringraziamenti della tesi).

Il quarto posto della top 3 (sono miei i ringraziamenti, quindi mie sono le regole) va a Lorena. Con te anche potrei sbrodolare un'altra tesi intera, un po\curlyapostrophe{} come sbrodolo le case a Capodanno. Davo per scontate molte cose di me, forse troppe. Ma in questi ultimi anni tu mi hai demolito quasi del tutto (o forse ci siamo demolite insieme), e poi ricostruito. Non potrò mai ringraziarti abbastanza. La tua sagacia, le tue domande scomode, le tue ampie vedute e le tue fragilità, tra le altre cose, sono ciò che più apprezzo di te. Non credevo che si potesse mai crescere così tanto insieme ad una persona. Spero che continueremo ad andare\ldots{} Molto Molto Lontano.

Vale e Luci, non riesco a pensarvi separate nella vita, e così non me la sono sentita di separarvi nemmeno qui. Quella del mio dottorato è solo un\curlyapostrophe{}altra esperienza che si aggiunge a tutte quelle che condividiamo ormai dal lontano 2000. Sembra incredibile a pensarci, ma siete probabilmente le persone che conosco da più tempo dopo la mia famiglia. Crescendo, siamo finitx a vivere distanti e a fare esperienze a volte anche molto diverse, ma la vostra capacità di far tornare tutto a essere così vicino e familiare vi rende per me la zona sicura in cui è sempre un piacere (e a volte un sollievo) tornare.

Ama Miki, sei una delle persone con cui mi trovo meglio a condividere i miei disagi e le mie conquiste, e credo di aver trovato in pochissime altre persone il tuo modo di saper ascoltare e di avere sempre il consiglio pronto. In quelle volte che mi è capitato di confrontarmi con te, che sia stato con una pizza in spiaggia, con un kebab di Babilionia o di fronte a una bruschetta con burro e alici (sì, credo che ci troviamo spesso per mangiare), poi mi sono sentito sempre meglio, compreso e un po\curlyapostrophe{} più rinvogorito. In ogni caso, lo sappiamo tuttx chi è chi, quindi non c\curlyapostrophe{}è bisogno di metterlo per iscritto nero su bianco, sarebbe uno spreco di inchiostro.

Helena e Clizia, so che ancora non sono mai venuto a Sirolo al mare con voi, e so anche che non abbiamo più giocato a cappello da quel fantomatico Ferragosto. Giuro che prima o poi recupererò. Ma almeno posso dire che nessun altrx mi ha mai accolto nel proprio gruppo di amicx così come avete fatto voi. Non so in che modo, ma nelle (poche) volte che ci si vede riuscite sempre a colmare i periodi di lontanza in un istante.

Non può mancare da questi ringraziamenti il resto della mia famiglia, inclusa ovviamente quella allargata. È un po\curlyapostrophe{} impegnativo starvi dietro a tuttx, ma forse è proprio questo che vi rende speciali. Non perdete mai occasione di strapparmi un sorriso o di creare bei ricordi. Una menzione d\curlyapostrophe{}onore va a Cami, con cui ho condiviso ansie e felicità dei rispettivi dottorati, tanti meme, una casa e dei gatti non nostri. 

Amicx del gruppo Euggin e affini, grazie. Grazie per aver reso i miei anni a Bologna così indimenticabili, densi di esperienze ed emozioni. Grazie per aver creato intorno a voi questa atmosfera di accoglienza, spensieratezza e genuinità che vi contraddistingue. Avete sicuramente contribuito a rendere la mia vita universitaria migliore, anche quando i pranzi delle 12 finivano per essere il caffè delle 15, la merenda delle 17, l\curlyapostrophe{}aperitivo delle 18, e che non prendiamo una pizza alle 20?, ma anche il bombocrep delle 23 e l\curlyapostrophe{}amaro dell\curlyapostrophe{}1.

A tutte le persone del secondo (e del primo) piano di via Selmi 2, grazie per aver creato l\curlyapostrophe{}ennesimo gruppo di cui mi sento indissolubilmente parte. Con voi ho condiviso davvero gioie e dolori di ogni tipo, ma le estenuanti e infinite settimane ai congressi saranno quelle che ricorderò con più piacere. Grazie per essere statx sempre un sostegno quando c\curlyapostrophe{}era bisogno, che fosse personale, lavorativo o non so di che altro tipo. A tal proposito, mi sento di ringraziare in particolare Giobbe, Mari e Giova, per l'enorme bontà d\curlyapostrophe{}animo che mettete in tutto ciò che fate e per essere statx dei veri riferimenti. Fabrizio e Liliana, la passione che mostrate per il vostro lavoro è una di quelle che di più mi ha ispirato mentre ero studente, e se oggi sono qui a scrivere queste righe è sicuramente anche merito vostro. 

A tutte le persone che ho incontrato a Los Angeles, sappiate che per me quella è stata una delle esperienze più bizzarre della mia vita, a posteriori. Era come vivere in un mondo parallelo, sospeso in un qualche limbo, in cui tutto era legittimo: dalle sessioni in laboratorio in orari improponibili, ai viaggi domenicali improvvisati, fino alle cene con il \qty{98}{\percent} di sconosciuti (chi mi conosce, sa che per me queste sono tutte cose che normalmente non mi verrebbe naturale fare). E quindi eccomi qui a ringraziarvi per avermele fatte fare: grazie alle persone del Nuzhdin Lab, soprattutto a Bernie, Inessa e Rachel, e grazie a quell\curlyapostrophe{}improbabile gruppo di persone nato in quell\curlyapostrophe{}altrettanto improbabile Oktoberfest californiano. Grazie a Sergey per averci creduto sin dall\curlyapostrophe{}inizio (o quasi) più di me.

Colleghx del 37\textdegree{} ciclo, finalmente è finita. C\curlyapostrophe{}era sempre qualcosa di cui lamentarsi, tra ricerca, burocrazia o intralci vari. E ogni volta c\curlyapostrophe{}eravate sempre per darci corda l\curlyapostrophe{}un l\curlyapostrophe{}altrx. Grazie per le settimane a Fano e per tutto ciò che abbiamo condiviso.

Ovviamente non sono riuscito ad essere poi tanto breve, ma poteva andare peggio\ldots{}
\end{otherlanguage*}