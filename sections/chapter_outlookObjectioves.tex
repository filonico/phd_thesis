% \documentclass[../main.tex]{subfiles}

% \begin{document}

{
\setstretch{1.0}
\chapter{Project outlook and objectives}
\label{chapter:outlookObjectives}
}

This PhD project focuses on understanding the evolutionary dynamics and molecular mechanisms underlying \gls{sd} in bivalve molluscs. The research has leveraged a wide array of analytical tools—from comparative genomics, to transcriptomics, \textit{in-situ} hybridization, and immunolocalization, in order to investigate \glspl{srg} across various species with an integrative and comparative approach. Particularly, special attention was given to the \gls{dsfg} families, which are widely-recognised key actors in the \gls{sd} process of the majority of animal species, including bivalves. Each major area of analysis in my research, together with its objectives, is presented in a dedicated chapter, resulting in three distinct sections.

\cref{chapter:perspective}, which consists of a perspective piece published in \textit{Genome Biology and Evolution}, will examine bivalves as emerging model organisms in \gls{sd} research, by reviewing their genomic and biological characteristics. Bivalve offers valuable insights into several topics, including
\begin{inlinelist}[itemjoin={{, }}, itemjoin*={{, and }}]
    \item the transitions between environmental and genetic \gls{sd}
    \item the evolution of \glspl{sc}
    \item the tentatively interaction between mitochondrial inheritance and \gls{sd}
    \item the evolutionary history of \glspl{srg}.
\end{inlinelist}    
Particularly, this chapter wants to emphasise the importance of establishing a comprehensive evolutionary genomics framework for studying \gls{sd} across bivalve species.

\cref{chapter:molecularEvolution} will explore the molecular evolution of some key \glspl{srg}. Using a broad genomic context that includes more than 40 annotated bivalve genomes and transcriptomes, this chapter aims to uncover how these genes have evolved and their potential roles in \gls{sd}, by also adopting a cross-species validation assay. The analysis will focus on the evolution of the \gls{dsfg} families, by using the tools of molecular evolution to assess whether some of them are tightly involved in \gls{sd}. Mammals and \textit{Drosophila} spp. will be used as positive and negative control datasets, respectively, to validate the reliability of the approach.

\cref{chapter:insitu} will focus on the expression patterns of three \glspl{srg} in the Mediterranean mussel \gls{mgal} during early developmental stages. Particularly, the spatial and temporal transcription patterns of \gls{dmrt-1l}, \textit{Sox-H}, and \textit{Fox-L2}---which have been identified as tightly linked to primary \gls{sd} by analyses in \cref{chapter:molecularEvolution} and in previous works, will be investigated. By also including the analysis of the germline marker \textit{Vasa}/Vasa, this chapter will provide novel insights into the mechanisms of \gls{sd} and \gls{pgc} specification. Transcription patterns will be investigated through computational \gls{dge} analyses and mRNA \textit{in-situ} \gls{hcr}; the expression pattern of \textit{Vasa}/Vasa will be investigated also through immunolocalization.

Overall, this PhD project aims to adopt a multi-layered and integrative approach that combines evolutionary genomics, gene expression analyses, and comparative biology to explore \gls{sd} in bivalves. Bivalves represent a relatively underexplored group, and given the remarkable diversity of their \gls{sd} processes, require a strong evolutionary perspective to decipher the mechanism. Here, the integration of genome-wide molecular evolution analysis with gene expression studies provides a novel framework for understanding how \glspl{srg}, such as those belonging to the \gls{dsfg} families, contribute to \gls{sd} and sexual differentiation. This work also benefits from cross-species comparisons, which places bivalve \gls{sd} within a broader evolutionary context, allowing for the identification of commonalities and unique traits in sex-determining pathways across taxa. Moreover, by investigating the expression patterns of three \glspl{srg} during early development in \gls{mgal}, this project addresses a critical gap in the understanding of how these genes may regulate the sexual process, as to date bivalve \gls{sd} has been investigated mostly in adult life stages. Through a comprehensive and comparative methodology, the project promises to provide a first reference broad-scale evolutionary resource for bivalve \gls{sd}, also pushing forward the boundaries of reproductive and evolutionary biology in non-model species.

% end{document}