% \documentclass[../main.tex]{subfiles}

% \begin{document}

{
\setstretch{1.0}
\chapter{Introduction}

\label{introduction}
}

\section{The diversity of sexual processes in animals}
The process of \gls{sd} has been traditionally associated with the very first steps of gonad differentiation, where an initial trigger (or master switch) activates the molecular pathway that establishes the sexual identity of an organism. According to this view, two alternative types of \gls{sd} have been traditionally recognised: the \gls{gsd} and the \gls{esd}, depending on whether the very first cues are of genetic or environmental origin. All the downstream events of gonad and morphological sex-specific development (i.e., after \gls{sd}) have been instead appointed as \gls{sdf}, which consists of the entire set of morphogenetic, molecular, and physiological events leading to the full maturation of testes or ovaries and secondary sexual characters (\citebold{uller2011origin,bear2013both,beukeboom2014evolution}). \gls{gsd} is found in many classical model systems---such as the mouse \gls{mmus}, the fruit fly \gls{dmel}, and the nematode \gls{cele}, but is by far the most prevalent \gls{sd} system in animals, as it occurs in the majority of vertebrates and arthropods (\citebold{bachtrog2014sex,beukeboom2014evolution}). It encompasses a variety of sex-determining cues, ranging from single master genes (e.g., \gls{sry} in eutherians), to polygenic networks (as in the zebrafish \gls{drer}) and chromosome countings (as in \gls{dmel} and \gls{cele}). Conversely, \gls{esd} is more uncommon and is found mainly in reptiles, fishes, insects, crustaceans, annelids, nematodes, and rotifers (reviewed in \citebold{korpelainen1990sex,bachtrog2014sex}). It relies on several initial stimuli of environmental origin, such as light, food availability, and population density, though the most common process is the \glsunset{tsd}\glsxtrlong{tsd} (\glsxtrshort{tsd}; \citebold{bachtrog2014sex,beukeboom2014evolution}). Currently, the molecular basis by which an environmental signal is transduced into the canalization of the male or female developmental pathway is unknown (\citebold{bachtrog2014sex,capel2017vertebrate}).

Lately, a growing number of studies have challenged the traditionally binary views of both \gls{gsd}/\gls{esd} and \gls{sd}/\gls{sdf} (\citebold{bear2013both,bachtrog2014sex,beukeboom2014evolution,todd2016bending,capel2017vertebrate}). On the one hand, the characterisation of \gls{sd} in new species has shown that \gls{gsd} and \gls{esd} represent the ends of a continuum of mixed conditions, rather than two mutually exclusive phenomena. For example, in the red-eared slider turtle \gls{tscr}, a species with \gls{tsd}, it has been shown that pairs of gonads cultured separately at the same pivotal temperature (i.e., the temperature producing \qty{50}{\percent} of males and \qty{50}{\percent} of females in offspring) tend to predominantly differentiate into one sex. Therefore, an underlying genetic/epigenetic mechanism controlling \gls{sd} should exist when temperature effect is absent (\citebold{mork2014predetermination}). In the Australian bearded-dragon \gls{pvit}, some ZZ males were caught to sex-revert to fertile females in the wild after incubation at high temperatures, constituting a natural case of transition from \gls{gsd} to \gls{esd} (in the form of \gls{tsd}; \citebold{holleley2015sex}). On the other hand, instead, the high evolutionary dynamics and the variable expression patterns of the genes involved in the processes of gonad commitment and development make the distinction between \gls{sd} and \gls{sdf} increasingly blurred (\citebold{bear2013both,bachtrog2014sex,beukeboom2014evolution,todd2016bending,capel2017vertebrate}). In fact, considering that the \gls{sd} of an organism may follow different routes, such as being triggered only in presumptive gonads---which then establish the sexual identity of the rest of the organism through hormone signalling (gonadal \pac{sd}; a process traditionally associated to mammals), or occurring independently in every cell of the developing embryo (cell-autonomous \pac{sd}; a process traditionally associated to fruit flies and nematodes), the tempo and modes of \gls{sdf} may vary significantly among species and may not even necessarily depend on or originate from \gls{sd} itself (\citebold{bear2013both,capel2017vertebrate}). Classic examples of the independence between \gls{sd} and \gls{sdf} are provided by gynandromorph animals (where both the male and the female phenotypes are found in the same organism in a bilaterally-distributed fashion), but also by the marsupial mammal tammar wallaby (\textit{Macropus} [now \textit{Notamacropus}] {eugenii}). In gynandromorph chickens (\gls{ggal}), for example, it has been shown that the male half of the animal is made up mainly by ZZ cells, while the female half by ZW cells, and that each is selectively susceptible to either male or female gonad-secreted hormones, respectively. This shows how in birds \gls{sd}, which is initiated in the presumptive gonads, is independent from and occurs later then \gls{sdf}, which is instead triggered by a cell-autonomous mechanism (\citebold{zhao2010somatic}). Similarly, in \textit{N. eugenii} the somatic differentiation of the scrotum and the pouch precedes that of the gonads, indicating that their development (as part of \gls{sdf}) is triggered by genes linked to the X chromosome, rather than by sex-differentiation inducing hormones as in other mammals (\citebold{renfree1996sexual}).

Considering this complex scenario, \citeboldyearparent{uller2011origin} proposed a unified and broad-scope definition for \gls{sd}, that is, \doublecurlyquotes{the processes within an embryo leading to the formation of differentiated gonads as either testes or ovaries}, thus excluding any actual distinction between environmental/genetic initial triggers or the downstream effectors. However, it can be argued that this definition should be even more expanded to encompass not only the embryonic stage of the animal life cycle but also adulthood, since cases of sex reversals (sequential hermaphroditism) legitimately express proper \gls{sd} processes also during post-embryonic life stages. For example, fishes represent a noteworthy example of how the establishment of sexual fate is not an irreversible process in an organism\curlyapostrophe s life, but instead a tradeoff trait, involving antagonistic \glspl{grn}, that can be switched from one side to the other (\citebold{todd2016bending,capel2017vertebrate}). Sex reversal in fishes is typically prompted by environmental signals, such as population density, sex ratio at spawning, and social factors, but also by the attainment of a threshold size and/or age (reviewed in \citebold{todd2016bending}). Regardless of its biology and direction (i.e., from males to females [protandry], from females to males [protogyny], or both ways), sex reversal often results in the complete restructuring of gonads, as well as in remarkable changes in both somatic morphology and behaviour. Therefore, sequential hermaphroditism can be considered the expression of a genuine \gls{sd} program inducing a second round of \gls{sdf} in adult organisms (\citebold{todd2016bending}).

Overall, decades of studies have revealed that \gls{sd} is strikingly diverse among animals, even in closely related species. Therefore, the research effort is currently devoted to further characterising sex-determining processes in new species, as well as to understand how this fundamental aspect of animal development may sustain such a high plasticity among species (\citebold{beukeboom2014evolution,todd2016bending,capel2017vertebrate}).

\section[Genetic sex determination and the evolution of sex-determination related genes]{Genetic sex determination and the evolution of \\ sex-determination related genes}\label{section:introduction-srg}
In its most intimate core, animal \gls{sd} is the manifestation of complex \glspl{grn} where, according to \citeboldyearparent{wilkins1995moving}, the downstream actors appear to be nearly conserved across species, while the master top regulators (the commonly recognized sex determinants, such as the \gls{sry} in therians or the ratio between sex and autosomal chromosomes in \textit{Drosophila}) are often the most variable part (\citebold{matson2012sex,mullon2012drosophila_sxl,bachtrog2014sex,beukeboom2014evolution}). Such a differential pattern of molecular evolution is considered to be the direct result of the mechanism by which a sex-determining cascade is assembled. Particularly, it has been proposed that a \gls{grn} may take on a role in \gls{sd} through a retrograde growth, i.e., by progressively adding upper regulators in a bottom-up process (\citebold{stothard2003sex,mullon2012drosophila_sxl,capel2017vertebrate}). This mechanism regards the \gls{sd} cascade in \textit{Drosophila} species (\citebold{mullon2012drosophila_sxl}), \gls{cele} (\citebold{stothard2003sex}), and vertebrates, although the latter case has been questioned several times (reviewed in \citebold{capel2017vertebrate}). Remarkably, it appears that some gene families are more prone than others to be recruited in \gls{sd}, as either primary \glspl{sdg} or in some key part of the cascade. For example, components of the \gls{dmrt} gene family have a main role as bottom effectors in the \gls{sd} cascade of many animal species, as seen in \gls{dmel} with \gls{dsx} and in \gls{cele} with \gls{mab-3}, but also in other invertebrates and the majority of vertebrates. In these cases, \gls{dmrt} genes dictate the sex-specific development in response to the primary \gls{sd} decision (\citebold{matson2012sex}). Nevertheless, paralogs of \textit{Dmrt-1} have also repeatedly and independently taken on the role as \glspl{sdg} in several vertebrate species, as in the medaka fish \gls{olat} with \gls{dmy}, in the African clawed frog \gls{xlae} with \gls{dmw}, and in \gls{ggal} with the Z-linked \textit{Dmrt-1} (reviewed in \citebold{matson2012sex,mawaribuchi2019independent}). A similar but even more conserved sex-determing genetic axis is found in insects, where \gls{tra} directs the sex-specific splicing of \gls{dsx} in almost every species investigated so far (\citebold{verhulst2010insect,bopp2014sex}). The \gls{grn} in which the \gls{tra}-\gls{dsx} module is placed, is instead more diversified and species-specific, as the top- and bottom-most parts are highly divergent, resulting in a \gls{sd} cascade that can be represented by an hour-glass model (\citebold{bopp2014sex}). Similarly, other highly-conserved genes involved in \gls{sd} has been identified, particularly as downstream effectors in vertebrates: these includes for example \textit{Fox-L2} from the \gls{fox} gene family and \textit{Sox-9} from the \gls{sox} gene families, acting in the female- and male-specific cascades, respectively (\citebold{capel2017vertebrate}).

The significance of molecular evolution in shaping \glspl{sdg} is also evident in the wider category of \glspl{srg}, which includes all the genes that are responsible for the specification, development and maintenance of the sexual identity. For example, transcriptionally sex-biased genes often tend to evolve faster than unbiased genes at the level of protein sequences. In particular, male-biased genes generally show higher rate of sequence evolution in comparison to both female-biased and unbiased counterparts, as it has been repeatedly observed in well-studied organisms---such as fruit flies, nematodes, mice and primates (reviewed in \citebold{parsch2013evolutionary,grath2016sex}), but also in other emerging systems, such as the water flea \gls{dpul} (\citebold{eads2007profiling}), aphids (\citebold{purandare2014accelerated}), and two wasp species of the genus \textit{Nasonia} (\citebold{wang2015nasonia}). That said, growing evidence is also showing cases in which female-biased genes have higher rates of sequence evolution than male-biased genes, such as in mosquitoes of the genus \textit{Anopheles} (\citebold{papa2017anopheles}), and European and Manila clams of the genus \textit{Ruditapes} (\citebold{ghiselli2018comparative}). High rates of molecular evolution in \glspl{srg} is particularly evident in organisms with \glspl{sc}---both in XY/ZW and X0 systems, such as fruit flies, birds and mammals, where the so-called fast-X (or fast-Z) effect has been extensively reported (\citebold{vicoso2006evolutionXchrom,mank2007fastZ,meisel2013faster}). In these species, accelerated sequence evolution is seen in general for genes residing on the X (or Z) chromosomes (i.e., the chromosomes determining the homogametic sex) with respect to genes of the autosomal chromosomes, and it could be explained by both adaptive and non-adaptive processes. In fact, the higher ratio of non-synonymous to synonymous mutations ($dN/dS$, or $\omega$) may result from positive selection, driven either by natural or sexual selection (as in \textit{Drosophila}), as well as from genetic drift (as in birds; \citebold{vicoso2006evolutionXchrom,meisel2013faster,parsch2013evolutionary,grath2016sex}).

\section{Unravelling sex determination in bivalves}
Bivalves are the second largest clade in molluscs, counting more than \noexponentnum{23000} species (\ulhref{https://www.catalogueoflife.org/data/taxon/BMGWC}{Catalogue of Life}; accessed on \DTMdate{2024-10-15}) distributed at all depths and in all marine environments, as well as in some freshwater habitats. Thanks to their high diversity and biological peculiarities, they have been proposed as promising model organisms for investigating a wide array of biological, ecological and evolutionary issues (\citebold{milani2020faraway,ghiselli2021bivalve,nicolini2023bivalves}). However, despite the socio-economic and scientific importance, the knowledge concerning the molecular basis of bivalve reproduction and \gls{sd} is still quite limited (\citebold{breton2018sex,nicolini2023bivalves}). Clues from various works seem to suggest that both genetic and environmental factors are involved in \gls{sd}, though the exact process by which sex is determined and gonad commitment is established is, currently, still unknown.

In the attempt to identify \glspl{srg} (including \glspl{sdg}), and clarify whether a single genetic determinant or a parliamentary decision exist, several \gls{dge} analyses have been recently performed on a variety of species (e.g., \citebold{milani2013nuclear,teaniniuraitemoana2014gonad,zhang2014genomic,chen2017transcriptome,capt2018deciphering,ghiselli2018comparative,shi2018proteome}). Particularly, some of the genes that were found to be differentially expressed between gonads of different sex were systematically retrieved, such as those belonging to the \gls{dsfg} families. To this regard, \citeboldyearparent{zhang2014genomic} proposed a working model for the sex-determining pathway of the Pacific oyster \gls{cgig} in which: \textit{Sox-H} promotes male gonad development by activating \gls{dmrt-1l}, and inhibiting \textit{Fox-L2}; \textit{Fox-L2}, when not inhibited by the pair \textit{Sox-H}/\gls{dmrt-1l}, promotes the female gonad development. Additionally, \textit{Fox-L2} has been appointed as the female \gls{sdg}---following a ZW inheritance system, in \gls{pyes} and \gls{cfar}, based on the analysis of read coverage and of the distribution of sexually dimorphic single-nucleotide polymorphisms (SNPs; \citebold{han2022ancient}). However, both the \gls{sd} model in \gls{cgig} and the role of \textit{Fox-L2} as the female \gls{sdg} in \gls{pyes} and \gls{cfar}, have never been fully tested from a functional point of view (e.g., through gene editing or knock-down), and thus remain only hypothesis. Overall, much of the recent research effort on bivalve \glspl{srg} (including \glspl{dsfg}) has indeed been limited to their molecular cloning, differential transcription, and tissue localization (\citebold{liang2019sox2,sun2022examination}), and few works have directly investigated the biological functions so far, mostly through post-transcriptional silencing of target mRNAs (\pac{rnai}). For example, \citeboldyearparent{liang2019sox2} studied the role of \textit{Sox2} in the spermatogenesis of the Zhikong scallop \gls{cfar} and found that it likely regulates proliferation of spermatogonia and apoptosis of spermatocytes, since its knockdown resulted in the loss of male germ cells. \citeboldyearparent{wang2020identification} proposed that in the female gonads of the freshwater mussel \gls{hcum}, \textit{Fox-L2} might be related to the \textit{Wnt}/\textit{$\beta$-catenin} signalling pathway, which takes part in ovarian differentiation also in vertebrates. \citeboldyearparent{sun2022examination} found instead that in \gls{cgig}, \textit{Fox-L2} and \gls{dmrt-1l} mRNA knockdown results in the size reduction of female and male mature gonads, respectively. The challenge in identifying \glspl{sdg}, if they exist, is partly due also to the apparent lack of \glspl{hesc} in all the bivalve species investigated to date (\citebold{breton2018sex,han2022ancient}). In fact, any evidence of \glspl{sc} has only been found in four scallop species (\textit{Amusium japonicum}, \gls{cfar}, \textit{Placopecten magellanicus}, \gls{pyes}), where they have been described as \glsunset{hosc}\glsxtrlongpl{hosc} (\glsxtrshortpl{hosc}; \citebold{han2022ancient}). Though, considering that \glspl{dsfg} generally work in a coordinated manner to regulate many developmental processes also in other animal species, including the \gls{sd} cascade itself (see \cref{section:introduction-srg}), it is reasonable to assume that they play similar roles also in bivalves.

Our understanding of the environmental influences on \gls{sd} is possibly even more limited. Given that bivalves exhibit a wide array of reproductive strategies—ranging from strict gonochorism to sequential (either protandrous or protogynous) and simultaneous hermaphroditism, as well as the so-called \singlecurlyquotes{alternative} and \singlecurlyquotes{rhythmical sexuality} (reviewed in \citebold{breton2018sex}), they represent an excellent model to investigate the mechanisms of \gls{esd}. Temperature, food availability, social factors, and xenobiotics all seem to influence \gls{sd}, or at least to trigger sex reversal in several hermaphroditic species (mainly belonging to the Ostreida and Pectinida orders). As a matter of fact, \gls{esd} has been investigated only in adult individuals through sex-ratio studies, thus few or no experiments are available for the very first round of \gls{sd} (i.e., that encompassing the first gonad specification cycle). The Pacific oyster \gls{cgig}, along with other oyster species, is one of the most studied bivalves not only for \gls{gsd} (as mentioned above), but also for \gls{esd}. It has been shown that the sex ratio of adults is influenced by the incubation temperature of immature spats: at \qty{18}{\degreeCelsius}, the sex ratio is skewed towards females, while at \qty{28}{\degreeCelsius} it favours males, with some simultaneous hermaphrodites also observed (\citebold{santerre2013oyster}). Considering that \gls{sd} in \gls{cgig} may be also under genetic control (\citebold{santerre2013oyster,zhang2014genomic}; reviewed in \citebold{breton2018sex}), these observations contribute to the growing evidence that a mixture of different factors govern \gls{sd} in the species. A similar hypothesis of a mixed \gls{sd} system has been suggested also for other species, such as \textit{Crassostrea corteziensis} (\citebold{chavez2008prospective}), \textit{Pinctada margaritifera} (\citebold{teaniniuraitemoana2016effect}), and \gls{medu} (\citebold{dalpe2022influence}), although more insightful and thorough investigations are needed (\citebold{dalpe2022influence}).

Clearly, bivalves represent a dazzling example of how the traditional representation of sex as genetically- or environmentally-determined, as well as the distinction between \gls{sd} and \gls{sdf}, can no longer be assumed as strictly dichotomous. A multifactorial model, in which many genes and environmental cues act in concert to establish the sexual identity of the individual, seems to better explain the extreme diversity of bivalve \gls{sd} systems (\citebold{breton2018sex}). Nonetheless, much work still needs to be done, especially in the functional characterisation of the molecular ground plan. Functional assays employing \gls{rnai} and clustered regularly interspaced short palindromic repeats (CRISPR) and CRISPR-associated protein 9 (CRISPR-Cas9) techniques (e.g., \citebold{wang2020identification,sun2022examination,wang2022transcriptome}) are finally making their way into the study of bivalve biology and have been proved essential instruments also for the investigation of sex-related traits. However, very few works have made extensive use of the comparative and integrative approach in bivalve studies so far, which hampers the possibility to infer general patterns for such a vast and diverse class of organisms (\citebold{milani2020faraway}).

% \end{document}